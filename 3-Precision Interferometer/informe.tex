%%%%%%%%%%%%%%%%%%%%%%%%%%%%%%%%%%%%%%%%%
% University/School Laboratory Report
% LaTeX Template
% Version 3.1 (25/3/14)
%
% This template has been downloaded from:
% http://www.LaTeXTemplates.com
%
% Original author:
% Linux and Unix Users Group at Virginia Tech Wiki 
% (https://vtluug.org/wiki/Example_LaTeX_chem_lab_report)
%
% License:
% CC BY-NC-SA 3.0 (http://creativecommons.org/licenses/by-nc-sa/3.0/)
%
%%%%%%%%%%%%%%%%%%%%%%%%%%%%%%%%%%%%%%%%%

%----------------------------------------------------------------------------------------
%	PACKAGES AND DOCUMENT CONFIGURATIONS
%----------------------------------------------------------------------------------------

\documentclass{article}

\usepackage[utf8]{inputenc}
\usepackage[spanish]{babel}
\usepackage[version=3]{mhchem} % Package for chemical equation typesetting
\usepackage{siunitx} % Provides the \SI{}{} and \si{} command for typesetting SI units
\usepackage{graphicx} % Required for the inclusion of images
\usepackage{natbib} % Required to change bibliography style to APA
\usepackage{amsmath} % Required for some math elements 
\usepackage{physics}
\setlength\parindent{0pt} % Removes all indentation from paragraphs

\renewcommand{\labelenumi}{\alph{enumi}.} % Make numbering in the enumerate environment by letter rather than number (e.g. section 6)

%\usepackage{times} % Uncomment to use the Times New Roman font

%----------------------------------------------------------------------------------------
%	DOCUMENT INFORMATION
%----------------------------------------------------------------------------------------

\title{Interferometría\\ Laboratorio de Física Intermedia} % Title

\author{Diego \textsc{Berrocal}} % Author name

\date{\today} % Date for the report

\begin{document}

\maketitle % insert the title, author and date

\begin{center}
  \begin{tabular}{l r}
    integrantes: & diego berrocal chinchay\\ % partner names
    & jose castañeda\\
    instructores: & hector loro \\ & carmen eyzaguirre
  \end{tabular}
\end{center}

% if you wish to include an abstract, uncomment the lines below
% \begin{abstract}
%   abstract text
% \end{abstract}

% ----------------------------------------------------------------------------------------
% section 1
% ----------------------------------------------------------------------------------------

\section{Objetivo}

\begin{itemize}
\item medir la longitud de onda de la luz emitida por un láser
  mediante el método del interferómetro de michelson y el método de
  fabry-perot
\item calcular el índice de refracción del aire como función de la
  presión

\end{itemize}

% if you have more than one objective, uncomment the below:
% \begin{description}
% \item[first objective] \hfill \\
%   objective 1 text
% \item[second objective] \hfill \\
%   objective 2 text
% \end{description}

\subsection{Fundamento Teórico}
  \subsection[Interferometría]{Interferometría}
  \label{sec:interferometria}

  
  De acuerdo con el pricipio de superposición, la
  intensidad del campo $\va{e}$ en un punto del espacio, procedente de
  los campos separados $\va{e}_1, \va{e}_2, ...$ de varias fuentes
  contributivas es $\va{e} = \va{e}_1 + \va{e}_2 + ...$.
  \\
  La peturbación óptica varía en el tiempo da una velocidad muy rápida
  típicamente del orden de $10^{(14)}hz$, de forma que el campo real resulta
  ser una cantidad poco práctica de detectar, por lo tanto resulta mejor
  plantear el estudio de la interferencia recurriendo a la irradiancia, que
  que puede medirse directamente usando una gran variedad de sensores, por
  por ejemplo fotocélulas, emulsiones , fotograía o el ojo humano.
  \\
  Del análisis que va a llevarse a continuación, para simplificar el asunto
  considere dos fuentes puntuales ${S_1}$ y ${S_2}$ que emiten ondas monocromáticas
  en una misma frecuencia en un medio homogéneo. Sea su separación $a$ mucho
  mayor que $\lambda$. Colóquese los puntos de observación $P$ lo suficientemente lejos
  de las fuentes para que en $P$ los frentes de onda sean planas y
  considérese ondas lindamente polarizadas cuya forma es:
  \begin{equation}
    \label{eq:1}
    \va{E}_1(\va{r},t) = \va{E}_{01}\cos(\va{k}_1\dotproduct\va{r} - w + \epsilon_1)
  \end{equation}
  
  \begin{equation}
    \label{eq:2}
    \va{E}_2(\va{r},t) = \va{E}_{01}\cos(\va{k}_2\dotproduct\va{r} - w + \epsilon_2)
  \end{equation}
  
  Sabemos que la irradiancia es $I \alpha <\va{E}^3>_T$, proporcional al promedio temporal
  de la magnitud de la intensidad del campo al cuadrado así como:
  \begin{equation}
    \label{eq:3}
    \va{E}^2 = \va{E}\dotproduct\va{E} = ( \va{E}_1 + \va{E}_2 )\dotproduct( \va{E}_1 -  \va{E}_2 ) = \va{E}_1^2 + \va{E}_2^2 + 2\va{E}_1\dotproduct\va{E}_2 
  \end{equation}
  Tomando el promedio a ambos lados:
  $$ I = I_1 + I_2 + I_{12}$$
  Donde:
  \begin{equation}
    \label{eq:4}
    \begin{aligned}
      I_1& = \epsilon v <\va{E}_1^2>_T &\ \alpha & <\va{E}_1^2>_T \\
      I_2& = \epsilon v <\va{E}_2^2>_T &\ \alpha & <\va{E}_2^2>_T \\
      I_{12}& = \epsilon v 2\va{E}_1\dotproduct\va{E}_2 &\ \alpha &2\va{E}_1\dotproduct\va{E}_2\\
    \end{aligned}
  \end{equation}
  La última expresión $I_{12}$, se denomina término de interferencia.
  \\
  Para calcularlo:
  \begin{equation}
    \label{eq:5}
    \begin{aligned}
      \va{E}_1\dotproduct\va{E}_2 =& \va{E}_{01}\dotproduct\va{E}_{02}\cos(\va{k}_1\dotproduct\va{r} - w + \epsilon_1)\cos(\va{k}_2\dotproduct\va{r} - w + \epsilon_2) \\
      =& \va{E}_{01}\dotproduct\va{E}_{02}[\cos(\va{k}_1\dotproduct\va{r} + \epsilon_1)\cos(\omega t) + \sin(\va{k}_1\dotproduct\va{r} + \epsilon_1)\sin(\omega t)]
      \\ & [\cos(\va{k}_2\dotproduct\va{r} + \epsilon_2)\cos(\omega t) + \sin(\va{k}_2\dotproduct\va{r} + \epsilon_2)\sin(\omega t)]
    \end{aligned}
  \end{equation}

  Tomando el promedio temporal, se efectúa el producto y se toma en consideración que
  $<\cos^2\omega t> = 1/2, <\sin^2\omega t> = 1/2, <\sin\omega t><\cos\omega t> = 0$
  obteniéndose:
  \begin{equation}
    \label{eq:6}
    <\va{E}_1\dotproduct\va{E}_2> = \frac{\va{E}_{01}\dotproduct\va{E}_{02}\cos(\va{k}_1\dotproduct\va{r} + \epsilon_1 + \va{k}_2\dotproduct\va{r} + \epsilon_2) }{2}
  \end{equation}
  
  El término de interferencia, por lo tanto, es:
  \begin{equation}
    \label{eq:7}
    \begin{aligned}
      I_{12}\  \alpha\ &\va{E}_1\dotproduct\va{E}_2\cos(\delta)
      \\
      \delta =\ & \va{k}_1\dotproduct\va{r} - \va{k}_2\dotproduct\va{r} + \epsilon_1 + \epsilon_2\ \ \ \text{diferencia de fase}
    \end{aligned}
  \end{equation}
  
  Obsérvese que si $\va{E}_1\perp\va{E}_2 => I = I_1 + I_2$ donde tales estados ortogonales $\beta$ se combinarán para dar algún estado
  de nombre arbitrario, pero la distribución de densidad de flujo quedará
  inalterada
  \\
  Sea $\va{E}_1\parallel \va{E}_2$, en este caso, la irradiancia se calcula escalarmente así.
  \begin{equation}
    \label{eq:8}
    \cos(\theta)
  \end{equation}
  
  % ----------------------------------------------------------------------------------------
  % section 2
  % ----------------------------------------------------------------------------------------

  \section{experimental data}
\begin{tabular}{|l|l|}
  balance used & \#4\\
  magnesium from sample bottle & \#1
\end{tabular}
% ----------------------------------------------------------------------------------------
% section 3
% ----------------------------------------------------------------------------------------

\section{sample calculation}

\begin{tabular}{ll}
  & = \SI{2.18}{\gram}\\
  mass numberof oxygen & = \SI{2.18}{\gram} - \SI{1.31}{\gram}\\
  & = \SI{0.87}{\gram}
\end{tabular}

because of this reaction, the required ratio is the atomic weight of
magnesium: \SI{16.00}{\gram} of oxygen as experimental mass of mg:
experimental mass of oxygen or $\frac{x}{1.31}=\frac{16}{0.87}$ from
which, $m_{\ce{mg}} = 16.00 \times \frac{1.31}{0.87} = 24.1 =
\SI{24}{\gram\per\mole}$ (to two significant figures).

% ----------------------------------------------------------------------------------------
% section 4
% ----------------------------------------------------------------------------------------

\section{results and conclusions}

the atomic weight of magnesium is concluded to be
\SI{24}{\gram\per\mol}, as determined by the stoichiometry of its
chemical combination with oxygen. this result is in agreement with the
accepted value.

% \begin{figure}[h]
%   \begin{center}
%     \includegraphics[width=0.65\textwidth]{placeholder} % include the image placeholder.png
%     \caption{figure caption.}
%   \end{center}
% \end{figure}

% ----------------------------------------------------------------------------------------
% section 5
% ----------------------------------------------------------------------------------------

\section{discussion of experimental uncertainty}

the accepted value (periodic table) is \SI{24.3}{\gram\per\mole}
\cite{smith:2012qr}. the percentage discrepancy between the accepted
value and the result obtained here is 1.3\%. because only a single
measurement was made, it is not possible to calculate an estimated
standard deviation.

the most obvious source of experimental uncertainty is the limited
precision of the balance. other potential sources of experimental
uncertainty are: the reaction might not be complete; if not enough
time was allowed for total oxidation, less than complete oxidation of
the magnesium might have, in part, reacted with nitrogen in the air
(incorrect reaction); the magnesium oxide might have absorbed water
from the air, and thus weigh ``too much." because the result obtained
is close to the accepted value it is possible that some of these
experimental uncertainties have fortuitously cancelled one another.

% ----------------------------------------------------------------------------------------
% section 6
% ----------------------------------------------------------------------------------------

\section{answers to definitions}

\begin{enumerate}
  \begin{item}
    the \emph{atomic weight of an element} is the relative weight of
    one of its atoms compared to c-12 with a weight of
    12.0000000$\ldots$, hydrogen with a weight of 1.008, to oxygen
    with a weight of 16.00. atomic weight is also the average weight
    of all the atoms of that element as they occur in nature.
  \end{item}
  \begin{item}
    the \emph{units of atomic weight} are two-fold, with an identical
    numerical value. they are g/mole of atoms (or just g/mol) or
    amu/atom.
  \end{item}
  \begin{item}
    \emph{percentage discrepancy} between an accepted (literature)
    value and an experimental value is
    \begin{equation*}
      \frac{\mathrm{experimental\;result} - \mathrm{accepted\;result}}{\mathrm{accepted\;result}}
    \end{equation*}
  \end{item}
\end{enumerate}

% ----------------------------------------------------------------------------------------
% section 7
% ----------------------------------------------------------------------------------------

\section[hola que hace]{hola que hace}
\begin{equation}
  \label{eq1}
\end{equation}

% ----------------------------------------------------------------------------------------
% bibliography
% ----------------------------------------------------------------------------------------

\bibliographystyle{apalike}

\bibliography{sample}

% ----------------------------------------------------------------------------------------
\tableofcontents{}

\end{document}